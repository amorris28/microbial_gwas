Typically, microbial community structure-function research questions are framed
as Does microbial composition matter to ecosystem function? The answer to this
question is obviously "yes" given that microorganisms mediate many of the
processes of interest. The question then becomes does microbial community
variation within a functional group or overall matter to ecosystem function? 
Generally it is assumed that microbial composition is functionally irrelevant, 
though studies of environmental treatments, common gardens, and reciprocal
transplants have shown that different community compositions are correlated with
different functions. The challenge with these studies is separating the close
connection between environmental variation, community composition, and ecosystem
function. Microorganisms cannot be useful to ecosystem modeling efforts if there
is no identifiable variation in composition that consistently alters function
across space, time, and environmental gradients. The question "do we need to
include microorganisms in ecosystem models?" depends on identifying variation in
the community associated with function independent of the environment. If that's
true, then the next step in understanding or modeling these processes is 
identify what components of the community alter ecosystem function.

It has been demonstrated that both geographic and ecological distance can lead
to population differentiation. For geography, local genetic
drift and reduced gene flow due to geographic separation can lead to population
differentation (Wright 1943). In addition, ecological distance can lead to both selection
against migrants (Hendry 2004) and local selection on endemic populations to
generate variation. While
these processes have primarily been described for macrobial populations, the
same principles apply to microorganisms (Martiny et al. ). 

Genome-wide association studies are used to identify the genetic variation
underlying phenotypic traits in populations of organisms. A classic example is
identifying genomic regions associated with genetic diseases in human
populations. Generally there is a correspondence between disease states, shared
ancestry, and geographic proximity of populations. If the goal were only to
build a model that would
predict the distribution of disease states, then controlling for this covariance
structure would be unnecessary. However, investigators are specifically
interested in identify loci to use as disease indicators or to identify
mechanisms for the genetic basis of disease. Therefore, they use statistical
techniques to control for population stratification.



Outline

1. We don't know whether microbes are important for function
2. BEF and CAFE research has focused on terrestrial plant communities where
richness and composition can alter ecosystem function
3. Microbial communities are studied in different ways and so these approaches
don't apply. Instead, we try to correlate functional gene (lots of assumptions)
or Prokaryote diversity (few assumptions) with function, neither of which work
very well.
4. A more analogous problem is GWAS
5. GWAS studies demonstrate population strucutre can produce false positives
6. Things exhibit isolation by distance (Sewall Wright) including microbes
(Martiny), which produce structure
7. Therefore, we should consider population structure when 
8. Reframing the question

Things I still want to talk about:
- 
- GWAS studies have demonstrated population structure due to historical effects
or selection can produce false positives 
- A more analogous approach



